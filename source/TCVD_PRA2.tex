% Options for packages loaded elsewhere
\PassOptionsToPackage{unicode}{hyperref}
\PassOptionsToPackage{hyphens}{url}
%
\documentclass[
]{article}
\usepackage{amsmath,amssymb}
\usepackage{lmodern}
\usepackage{iftex}
\ifPDFTeX
  \usepackage[T1]{fontenc}
  \usepackage[utf8]{inputenc}
  \usepackage{textcomp} % provide euro and other symbols
\else % if luatex or xetex
  \usepackage{unicode-math}
  \defaultfontfeatures{Scale=MatchLowercase}
  \defaultfontfeatures[\rmfamily]{Ligatures=TeX,Scale=1}
\fi
% Use upquote if available, for straight quotes in verbatim environments
\IfFileExists{upquote.sty}{\usepackage{upquote}}{}
\IfFileExists{microtype.sty}{% use microtype if available
  \usepackage[]{microtype}
  \UseMicrotypeSet[protrusion]{basicmath} % disable protrusion for tt fonts
}{}
\makeatletter
\@ifundefined{KOMAClassName}{% if non-KOMA class
  \IfFileExists{parskip.sty}{%
    \usepackage{parskip}
  }{% else
    \setlength{\parindent}{0pt}
    \setlength{\parskip}{6pt plus 2pt minus 1pt}}
}{% if KOMA class
  \KOMAoptions{parskip=half}}
\makeatother
\usepackage{xcolor}
\usepackage[margin=1in]{geometry}
\usepackage{color}
\usepackage{fancyvrb}
\newcommand{\VerbBar}{|}
\newcommand{\VERB}{\Verb[commandchars=\\\{\}]}
\DefineVerbatimEnvironment{Highlighting}{Verbatim}{commandchars=\\\{\}}
% Add ',fontsize=\small' for more characters per line
\usepackage{framed}
\definecolor{shadecolor}{RGB}{248,248,248}
\newenvironment{Shaded}{\begin{snugshade}}{\end{snugshade}}
\newcommand{\AlertTok}[1]{\textcolor[rgb]{0.94,0.16,0.16}{#1}}
\newcommand{\AnnotationTok}[1]{\textcolor[rgb]{0.56,0.35,0.01}{\textbf{\textit{#1}}}}
\newcommand{\AttributeTok}[1]{\textcolor[rgb]{0.77,0.63,0.00}{#1}}
\newcommand{\BaseNTok}[1]{\textcolor[rgb]{0.00,0.00,0.81}{#1}}
\newcommand{\BuiltInTok}[1]{#1}
\newcommand{\CharTok}[1]{\textcolor[rgb]{0.31,0.60,0.02}{#1}}
\newcommand{\CommentTok}[1]{\textcolor[rgb]{0.56,0.35,0.01}{\textit{#1}}}
\newcommand{\CommentVarTok}[1]{\textcolor[rgb]{0.56,0.35,0.01}{\textbf{\textit{#1}}}}
\newcommand{\ConstantTok}[1]{\textcolor[rgb]{0.00,0.00,0.00}{#1}}
\newcommand{\ControlFlowTok}[1]{\textcolor[rgb]{0.13,0.29,0.53}{\textbf{#1}}}
\newcommand{\DataTypeTok}[1]{\textcolor[rgb]{0.13,0.29,0.53}{#1}}
\newcommand{\DecValTok}[1]{\textcolor[rgb]{0.00,0.00,0.81}{#1}}
\newcommand{\DocumentationTok}[1]{\textcolor[rgb]{0.56,0.35,0.01}{\textbf{\textit{#1}}}}
\newcommand{\ErrorTok}[1]{\textcolor[rgb]{0.64,0.00,0.00}{\textbf{#1}}}
\newcommand{\ExtensionTok}[1]{#1}
\newcommand{\FloatTok}[1]{\textcolor[rgb]{0.00,0.00,0.81}{#1}}
\newcommand{\FunctionTok}[1]{\textcolor[rgb]{0.00,0.00,0.00}{#1}}
\newcommand{\ImportTok}[1]{#1}
\newcommand{\InformationTok}[1]{\textcolor[rgb]{0.56,0.35,0.01}{\textbf{\textit{#1}}}}
\newcommand{\KeywordTok}[1]{\textcolor[rgb]{0.13,0.29,0.53}{\textbf{#1}}}
\newcommand{\NormalTok}[1]{#1}
\newcommand{\OperatorTok}[1]{\textcolor[rgb]{0.81,0.36,0.00}{\textbf{#1}}}
\newcommand{\OtherTok}[1]{\textcolor[rgb]{0.56,0.35,0.01}{#1}}
\newcommand{\PreprocessorTok}[1]{\textcolor[rgb]{0.56,0.35,0.01}{\textit{#1}}}
\newcommand{\RegionMarkerTok}[1]{#1}
\newcommand{\SpecialCharTok}[1]{\textcolor[rgb]{0.00,0.00,0.00}{#1}}
\newcommand{\SpecialStringTok}[1]{\textcolor[rgb]{0.31,0.60,0.02}{#1}}
\newcommand{\StringTok}[1]{\textcolor[rgb]{0.31,0.60,0.02}{#1}}
\newcommand{\VariableTok}[1]{\textcolor[rgb]{0.00,0.00,0.00}{#1}}
\newcommand{\VerbatimStringTok}[1]{\textcolor[rgb]{0.31,0.60,0.02}{#1}}
\newcommand{\WarningTok}[1]{\textcolor[rgb]{0.56,0.35,0.01}{\textbf{\textit{#1}}}}
\usepackage{graphicx}
\makeatletter
\def\maxwidth{\ifdim\Gin@nat@width>\linewidth\linewidth\else\Gin@nat@width\fi}
\def\maxheight{\ifdim\Gin@nat@height>\textheight\textheight\else\Gin@nat@height\fi}
\makeatother
% Scale images if necessary, so that they will not overflow the page
% margins by default, and it is still possible to overwrite the defaults
% using explicit options in \includegraphics[width, height, ...]{}
\setkeys{Gin}{width=\maxwidth,height=\maxheight,keepaspectratio}
% Set default figure placement to htbp
\makeatletter
\def\fps@figure{htbp}
\makeatother
\setlength{\emergencystretch}{3em} % prevent overfull lines
\providecommand{\tightlist}{%
  \setlength{\itemsep}{0pt}\setlength{\parskip}{0pt}}
\setcounter{secnumdepth}{5}
\newlength{\cslhangindent}
\setlength{\cslhangindent}{1.5em}
\newlength{\csllabelwidth}
\setlength{\csllabelwidth}{3em}
\newlength{\cslentryspacingunit} % times entry-spacing
\setlength{\cslentryspacingunit}{\parskip}
\newenvironment{CSLReferences}[2] % #1 hanging-ident, #2 entry spacing
 {% don't indent paragraphs
  \setlength{\parindent}{0pt}
  % turn on hanging indent if param 1 is 1
  \ifodd #1
  \let\oldpar\par
  \def\par{\hangindent=\cslhangindent\oldpar}
  \fi
  % set entry spacing
  \setlength{\parskip}{#2\cslentryspacingunit}
 }%
 {}
\usepackage{calc}
\newcommand{\CSLBlock}[1]{#1\hfill\break}
\newcommand{\CSLLeftMargin}[1]{\parbox[t]{\csllabelwidth}{#1}}
\newcommand{\CSLRightInline}[1]{\parbox[t]{\linewidth - \csllabelwidth}{#1}\break}
\newcommand{\CSLIndent}[1]{\hspace{\cslhangindent}#1}
\usepackage{booktabs}
\usepackage{longtable}
\usepackage{array}
\usepackage{multirow}
\usepackage{wrapfig}
\usepackage{float}
\usepackage{colortbl}
\usepackage{pdflscape}
\usepackage{tabu}
\usepackage{threeparttable}
\usepackage{threeparttablex}
\usepackage[normalem]{ulem}
\usepackage{makecell}
\usepackage{xcolor}
\ifLuaTeX
  \usepackage{selnolig}  % disable illegal ligatures
\fi
\IfFileExists{bookmark.sty}{\usepackage{bookmark}}{\usepackage{hyperref}}
\IfFileExists{xurl.sty}{\usepackage{xurl}}{} % add URL line breaks if available
\urlstyle{same} % disable monospaced font for URLs
\hypersetup{
  pdftitle={TCVD Práctica 2: ¿Cómo realizar la limpieza y análisis de datos?},
  pdfauthor={Alfonso Manuel Carvajal, Eider Ibiricu},
  hidelinks,
  pdfcreator={LaTeX via pandoc}}

\title{TCVD Práctica 2: ¿Cómo realizar la limpieza y análisis de datos?}
\author{Alfonso Manuel Carvajal, Eider Ibiricu}
\date{16 de June, 2023}

\begin{document}
\maketitle

\hypertarget{descripciuxf3n-del-dataset}{%
\section{Descripción del dataset}\label{descripciuxf3n-del-dataset}}

El dataset \emph{Heart Attack Analysis \& Prediction}
\href{https://www.kaggle.com/datasets/rashikrahmanpritom/heart-attack-analysis-prediction-dataset}{kaggle}
contiene datos para realizar una clasificación de pacientes que tengan
riesgo de sufrir un ataque al corazón.

Mediante este juego de datos es posible entrenar algoritmos que permitan
un diagnóstico para futuros posibles pacientes.

\begin{Shaded}
\begin{Highlighting}[]
\CommentTok{\#https://medium.com/mcd{-}unison/how{-}to{-}use{-}kaggle{-}api{-}to{-}download{-}datasets{-}in{-}r{-}312179c7a99c}

\FunctionTok{library}\NormalTok{(readr)}
\FunctionTok{library}\NormalTok{(kaggler)}
\FunctionTok{kgl\_auth}\NormalTok{(}\AttributeTok{creds\_file =} \StringTok{\textquotesingle{}kaggle.json\textquotesingle{}}\NormalTok{)}
\end{Highlighting}
\end{Shaded}

\begin{verbatim}
## <request>
## Options:
## * httpauth: 1
## * userpwd: eideribiricupera:2323ed93bf13610b7984e10597bedc72
\end{verbatim}

\begin{Shaded}
\begin{Highlighting}[]
\NormalTok{response }\OtherTok{\textless{}{-}} \FunctionTok{kgl\_datasets\_download\_all}\NormalTok{(}\AttributeTok{owner\_dataset =}
            \StringTok{"rashikrahmanpritom/heart{-}attack{-}analysis{-}prediction{-}dataset"}\NormalTok{)}

\FunctionTok{download.file}\NormalTok{(response[[}\StringTok{"url"}\NormalTok{]], }\StringTok{"data/temp.zip"}\NormalTok{, }\AttributeTok{mode=}\StringTok{"wb"}\NormalTok{)}
\NormalTok{unzip\_result }\OtherTok{\textless{}{-}} \FunctionTok{unzip}\NormalTok{(}\StringTok{"data/temp.zip"}\NormalTok{, }\AttributeTok{exdir =} \StringTok{"data/"}\NormalTok{, }\AttributeTok{overwrite =} \ConstantTok{TRUE}\NormalTok{)}
\NormalTok{unzip\_result}
\end{Highlighting}
\end{Shaded}

\begin{verbatim}
## [1] "data//heart.csv"        "data//o2Saturation.csv"
\end{verbatim}

\begin{Shaded}
\begin{Highlighting}[]
\NormalTok{heart\_attack\_data }\OtherTok{\textless{}{-}} \FunctionTok{read\_csv}\NormalTok{(}\StringTok{"data/heart.csv"}\NormalTok{)}
\NormalTok{o2\_saturation\_data }\OtherTok{\textless{}{-}} \FunctionTok{read.csv}\NormalTok{(}\StringTok{"data/o2Saturation.csv"}\NormalTok{,}\AttributeTok{header=}\NormalTok{F)}

\FunctionTok{rm}\NormalTok{(response)}
\end{Highlighting}
\end{Shaded}

Las variables que encontramos en el dataset, según la descripción en
\emph{kaggle}:

\begin{itemize}
\item
  \textbf{Age}: Edad del individuo. (Variable numérica continua)
\item
  \textbf{Sex}: Género del individuo (1 = masculino, 0 = femenino).
  (Variable categórica binaria)
\item
  \textbf{cp}: Tipo de dolor en el pecho (categorica ordinal)

  \begin{itemize}
  \item
    Value 1: typical angina
  \item
    Value 2: atypical angina
  \item
    Value 3: non-anginal pain
  \item
    Value 4: asymptomatic
  \end{itemize}
\item
  \textbf{trtbps}: Presión arterial en reposo (en mm Hg) (Variable
  numérica continua)
\item
  \textbf{chol}: Colesterol en mg/dl obtenido via sensor BMI (Variable
  numérica continua)
\item
  \textbf{fbs}: Nivel de azúcar en sangre en ayunas (\textgreater{} 120
  mg/dl, 1 = verdadero; 0 = falso). (Variable categórica binaria)
\item
  \textbf{restecg}: Resultados electrocardiográficos en reposo.
  (Variable categórica ordinal)

  \begin{itemize}
  \item
    Value 0: normal
  \item
    Value 1: having ST-T wave abnormality (T wave inversions and/or ST
    elevation or depression of \textgreater{} 0.05 mV)
  \item
    Value 2: showing probable or definite left ventricular hypertrophy
    by Estes' criteria
  \end{itemize}
\item
  \textbf{thalachh}: Máxima frecuencia cardíaca alcanzada. (Variable
  numérica continua)
\item
  \textbf{exng}: Angina inducida por ejercicio (1 = sí; 0 = no).
  (Variable categórica binaria)
\item
  \textbf{oldpeak}: Pico anterior, cambios en el segmento ST en un ECG
  (Variable numérica continua)
\item
  \textbf{slp}: La pendiente del segmento ST en el pico de ejercicio
  (Variable numérica continua)
\item
  \textbf{caa}: Número de vasos principales coloreados por fluoroscopia.
  (0-4) (categorical)
\item
  \textbf{thall}: Talio en sangre.(Thallium Stress Test )(numerica)
\item
  \textbf{output}: Diagnóstico de enfermedad cardíaca (estado del
  objetivo) (0 = menos probabilidad de ataque al corazón, 1 = más
  probabilidad de ataque al corazón). (Variable categórica binaria)
\end{itemize}

¿Por qué es importante y qué pregunta/problema pretende responder?

\hypertarget{integraciuxf3n-y-selecciuxf3n-de-los-datos-de-interuxe9s-a-analizar.}{%
\section{Integración y selección de los datos de interés a
analizar.}\label{integraciuxf3n-y-selecciuxf3n-de-los-datos-de-interuxe9s-a-analizar.}}

Puede ser el resultado de adicionar diferentes datasets o una
subselección útil de los datos originales, en base al objetivo que se
quiera conseguir.

La recopilación y elección de información constituyen etapas
fundamentales en cualquier tarea de análisis de datos. En relación a
nuestro conjunto de información, estos procesos significarán identificar
los factores que resultan más significativos para estimar un infarto al
miocardio y optar por aquellos que nos brinden la mayor utilidad en
nuestra investigación.

Dado que todas las variables en este conjunto de datos están
directamente vinculadas a la salud cardiovascular y los riesgos
asociados, todas podrían considerarse pertinentes. No obstante, puede
que no todas estas variables contribuyan de la misma manera a la
capacidad predictiva de un modelo de estimación de infartos al
miocardio.

Por ejemplo, las variables \texttt{age}, \texttt{sex}, \texttt{cp},
\texttt{trtbps}, \texttt{chol}, \texttt{fbs}, \texttt{restecg},
\texttt{thalachh}, \texttt{exng}, \texttt{oldpeak}, \texttt{slp},
\texttt{caa}, y \texttt{thall} son todos posibles factores de riesgo
para un infarto al miocardio y por lo tanto son de importancia para
nuestro estudio. La variable \texttt{output} es la que nos gustaría
pronosticar.

En consecuencia, el paso inicial en nuestro estudio será llevar a cabo
un examen exploratorio de los datos para comprender de mejor manera la
distribución y las relaciones de estas variables. Esto puede implicar
visualizar la distribución de la información, calcular estadísticas
descriptivas y analizar las correlaciones entre las diversas variables.

\begin{Shaded}
\begin{Highlighting}[]
\FunctionTok{head}\NormalTok{(heart\_attack\_data) }\SpecialCharTok{\%\textgreater{}\%} 
\NormalTok{  kable\_setup }\SpecialCharTok{\%\textgreater{}\%} 
  \FunctionTok{kable\_paper}\NormalTok{(}\AttributeTok{full\_width =}\NormalTok{ F)}\SpecialCharTok{\%\textgreater{}\%} 
  \FunctionTok{column\_spec}\NormalTok{(}\FunctionTok{c}\NormalTok{(}\DecValTok{3}\NormalTok{,}\DecValTok{7}\NormalTok{,}\DecValTok{14}\NormalTok{), }\AttributeTok{width =} \StringTok{"2 cm"}\NormalTok{) }\SpecialCharTok{\%\textgreater{}\%}
  \FunctionTok{column\_spec}\NormalTok{(}\FunctionTok{c}\NormalTok{(}\DecValTok{1}\NormalTok{,}\DecValTok{5}\NormalTok{,}\DecValTok{9}\NormalTok{,}\DecValTok{12}\NormalTok{,}\DecValTok{13}\NormalTok{), }\AttributeTok{width =} \StringTok{"0.8 cm"}\NormalTok{) }\SpecialCharTok{\%\textgreater{}\%}
  \FunctionTok{row\_spec}\NormalTok{(}\DecValTok{0}\NormalTok{,}\AttributeTok{bold=}\ConstantTok{TRUE}\NormalTok{)}
\end{Highlighting}
\end{Shaded}

\begin{table}[H]
\centering
\resizebox{\linewidth}{!}{
\begin{tabular}{>{\raggedleft\arraybackslash}p{0.8 cm}r>{\raggedleft\arraybackslash}p{2 cm}r>{\raggedleft\arraybackslash}p{0.8 cm}r>{\raggedleft\arraybackslash}p{2 cm}r>{\raggedleft\arraybackslash}p{0.8 cm}rr>{\raggedleft\arraybackslash}p{0.8 cm}>{\raggedleft\arraybackslash}p{0.8 cm}>{\raggedleft\arraybackslash}p{2 cm}}
\toprule
\textbf{age} & \textbf{sex} & \textbf{cp} & \textbf{trtbps} & \textbf{chol} & \textbf{fbs} & \textbf{restecg} & \textbf{thalachh} & \textbf{exng} & \textbf{oldpeak} & \textbf{slp} & \textbf{caa} & \textbf{thall} & \textbf{output}\\
\midrule
\cellcolor{gray!6}{63} & \cellcolor{gray!6}{1} & \cellcolor{gray!6}{3} & \cellcolor{gray!6}{145} & \cellcolor{gray!6}{233} & \cellcolor{gray!6}{1} & \cellcolor{gray!6}{0} & \cellcolor{gray!6}{150} & \cellcolor{gray!6}{0} & \cellcolor{gray!6}{2.3} & \cellcolor{gray!6}{0} & \cellcolor{gray!6}{0} & \cellcolor{gray!6}{1} & \cellcolor{gray!6}{1}\\
37 & 1 & 2 & 130 & 250 & 0 & 1 & 187 & 0 & 3.5 & 0 & 0 & 2 & 1\\
\cellcolor{gray!6}{41} & \cellcolor{gray!6}{0} & \cellcolor{gray!6}{1} & \cellcolor{gray!6}{130} & \cellcolor{gray!6}{204} & \cellcolor{gray!6}{0} & \cellcolor{gray!6}{0} & \cellcolor{gray!6}{172} & \cellcolor{gray!6}{0} & \cellcolor{gray!6}{1.4} & \cellcolor{gray!6}{2} & \cellcolor{gray!6}{0} & \cellcolor{gray!6}{2} & \cellcolor{gray!6}{1}\\
56 & 1 & 1 & 120 & 236 & 0 & 1 & 178 & 0 & 0.8 & 2 & 0 & 2 & 1\\
\cellcolor{gray!6}{57} & \cellcolor{gray!6}{0} & \cellcolor{gray!6}{0} & \cellcolor{gray!6}{120} & \cellcolor{gray!6}{354} & \cellcolor{gray!6}{0} & \cellcolor{gray!6}{1} & \cellcolor{gray!6}{163} & \cellcolor{gray!6}{1} & \cellcolor{gray!6}{0.6} & \cellcolor{gray!6}{2} & \cellcolor{gray!6}{0} & \cellcolor{gray!6}{2} & \cellcolor{gray!6}{1}\\
\addlinespace
57 & 1 & 0 & 140 & 192 & 0 & 1 & 148 & 0 & 0.4 & 1 & 0 & 1 & 1\\
\bottomrule
\end{tabular}}
\end{table}

\hypertarget{limpieza-de-los-datos}{%
\section{Limpieza de los datos}\label{limpieza-de-los-datos}}

Primero asignamos los tipos de datos a cada variable. En los
\textbf{\href{https://www.kaggle.com/datasets/rashikrahmanpritom/heart-attack-analysis-prediction-dataset/discussion/329925}{comentarios}}
del dataset, hemos encontrado definiciones de los campos que nos ayudan
a determinar el tipo de los datos:

\begin{Shaded}
\begin{Highlighting}[]
\CommentTok{\# Data types}

\NormalTok{data }\OtherTok{\textless{}{-}}\NormalTok{ heart\_attack\_data }\SpecialCharTok{\%\textgreater{}\%} 
  \FunctionTok{mutate}\NormalTok{(}
    \AttributeTok{sex =} \FunctionTok{factor}\NormalTok{(sex, }\AttributeTok{levels=}\FunctionTok{c}\NormalTok{(}\DecValTok{1}\NormalTok{,}\DecValTok{0}\NormalTok{),}\AttributeTok{labels =} \FunctionTok{c}\NormalTok{(}\StringTok{"male"}\NormalTok{,}\StringTok{"female"}\NormalTok{)),}
    \AttributeTok{cp =} \FunctionTok{factor}\NormalTok{(cp,}
                \AttributeTok{levels=}\FunctionTok{c}\NormalTok{(}\DecValTok{0}\NormalTok{,}\DecValTok{1}\NormalTok{,}\DecValTok{2}\NormalTok{,}\DecValTok{3}\NormalTok{),}
                \AttributeTok{labels=}\FunctionTok{c}\NormalTok{(}\StringTok{"typical angina"}\NormalTok{,}\StringTok{"atypical angina"}\NormalTok{,}\StringTok{"non{-}anginal pain"}\NormalTok{,}\StringTok{"asymptomatic"}\NormalTok{)),}
    \AttributeTok{fbs =} \FunctionTok{factor}\NormalTok{(fbs,}\AttributeTok{levels=}\FunctionTok{c}\NormalTok{(}\DecValTok{0}\NormalTok{,}\DecValTok{1}\NormalTok{),}\AttributeTok{labels =} \FunctionTok{c}\NormalTok{(F,T)),}
    \AttributeTok{restecg =} \FunctionTok{factor}\NormalTok{(restecg,}\AttributeTok{levels =} \FunctionTok{c}\NormalTok{(}\DecValTok{0}\NormalTok{,}\DecValTok{1}\NormalTok{,}\DecValTok{2}\NormalTok{),}
                     \AttributeTok{labels =} \FunctionTok{c}\NormalTok{(}\StringTok{"normal"}\NormalTok{,}\StringTok{"ST{-}T wave abnormality"}\NormalTok{,}\StringTok{"left ventricular hypertrophy"}\NormalTok{)),}
    \AttributeTok{exng =} \FunctionTok{factor}\NormalTok{(exng),}
    \AttributeTok{slp =} \FunctionTok{factor}\NormalTok{(slp,}\AttributeTok{levels =} \FunctionTok{c}\NormalTok{(}\DecValTok{0}\NormalTok{,}\DecValTok{1}\NormalTok{,}\DecValTok{2}\NormalTok{),}
                 \AttributeTok{labels =} \FunctionTok{c}\NormalTok{(}\StringTok{"unsloping"}\NormalTok{,}\StringTok{"flat"}\NormalTok{,}\StringTok{"downsloping"}\NormalTok{)),}
    \AttributeTok{caa =} \FunctionTok{factor}\NormalTok{(caa),}
    \AttributeTok{thall =} \FunctionTok{factor}\NormalTok{(thall, }
                   \AttributeTok{levels =} \FunctionTok{c}\NormalTok{(}\DecValTok{1}\NormalTok{,}\DecValTok{2}\NormalTok{,}\DecValTok{3}\NormalTok{),}
                   \AttributeTok{labels =} \FunctionTok{c}\NormalTok{(}\StringTok{"fixed defect"}\NormalTok{,}\StringTok{"normal"}\NormalTok{,}\StringTok{"reversable defect"}\NormalTok{)),}
    \AttributeTok{output =} \FunctionTok{factor}\NormalTok{(output,}\AttributeTok{levels=}\FunctionTok{c}\NormalTok{(}\DecValTok{0}\NormalTok{,}\DecValTok{1}\NormalTok{),}
                    \AttributeTok{labels =} \FunctionTok{c}\NormalTok{(}\StringTok{"less chance of heart attack"}\NormalTok{,}\StringTok{"more chance of heart attack"}\NormalTok{))}
\NormalTok{  )}
\end{Highlighting}
\end{Shaded}

Hacemos un resumen de los datos para identificar posibles valores nulos
o atípicos. Esta tabla también nos permite entender los rangos en los
que se mueven las variables.

También revisaremos si hay registros repetidos.

\begin{Shaded}
\begin{Highlighting}[]
\CommentTok{\# Summary}
\FunctionTok{summary}\NormalTok{(data) }
\end{Highlighting}
\end{Shaded}

\begin{verbatim}
##       age            sex                     cp          trtbps     
##  Min.   :29.00   male  :207   typical angina  :143   Min.   : 94.0  
##  1st Qu.:47.50   female: 96   atypical angina : 50   1st Qu.:120.0  
##  Median :55.00                non-anginal pain: 87   Median :130.0  
##  Mean   :54.37                asymptomatic    : 23   Mean   :131.6  
##  3rd Qu.:61.00                                       3rd Qu.:140.0  
##  Max.   :77.00                                       Max.   :200.0  
##       chol          fbs                              restecg       thalachh    
##  Min.   :126.0   FALSE:258   normal                      :147   Min.   : 71.0  
##  1st Qu.:211.0   TRUE : 45   ST-T wave abnormality       :152   1st Qu.:133.5  
##  Median :240.0               left ventricular hypertrophy:  4   Median :153.0  
##  Mean   :246.3                                                  Mean   :149.6  
##  3rd Qu.:274.5                                                  3rd Qu.:166.0  
##  Max.   :564.0                                                  Max.   :202.0  
##  exng       oldpeak              slp      caa                   thall    
##  0:204   Min.   :0.00   unsloping  : 21   0:175   fixed defect     : 18  
##  1: 99   1st Qu.:0.00   flat       :140   1: 65   normal           :166  
##          Median :0.80   downsloping:142   2: 38   reversable defect:117  
##          Mean   :1.04                     3: 20   NA's             :  2  
##          3rd Qu.:1.60                     4:  5                          
##          Max.   :6.20                                                    
##                          output   
##  less chance of heart attack:138  
##  more chance of heart attack:165  
##                                   
##                                   
##                                   
## 
\end{verbatim}

\begin{Shaded}
\begin{Highlighting}[]
\CommentTok{\# Duplicates}
\NormalTok{data }\SpecialCharTok{\%\textgreater{}\%} 
  \FunctionTok{unique}\NormalTok{() }\SpecialCharTok{\%\textgreater{}\%} 
  \FunctionTok{nrow}\NormalTok{()}
\end{Highlighting}
\end{Shaded}

\begin{verbatim}
## [1] 302
\end{verbatim}

\begin{Shaded}
\begin{Highlighting}[]
\NormalTok{data }\OtherTok{\textless{}{-}}\NormalTok{ data }\SpecialCharTok{\%\textgreater{}\%} 
  \FunctionTok{unique}\NormalTok{()}
\end{Highlighting}
\end{Shaded}

\hypertarget{los-datos-contienen-ceros-o-elementos-vacuxedos}{%
\subsection{¿Los datos contienen ceros o elementos
vacíos?}\label{los-datos-contienen-ceros-o-elementos-vacuxedos}}

Representamos las distribuciones de las variables:

\begin{center}\includegraphics{TCVD_PRA2_files/figure-latex/unnamed-chunk-4-1} \end{center}

En el caso de las variables numéricas

\begin{center}\includegraphics{TCVD_PRA2_files/figure-latex/unnamed-chunk-5-1} \end{center}

Vemos que la variable \emph{thall} tiene valores nulos. Podemos
considerar que estos registros como normales, por lo que les asignaremos
el valor 2:

\begin{Shaded}
\begin{Highlighting}[]
\NormalTok{data }\OtherTok{\textless{}{-}}\NormalTok{ data }\SpecialCharTok{\%\textgreater{}\%} 
  \FunctionTok{mutate}\NormalTok{(}
    \AttributeTok{thall =} \FunctionTok{factor}\NormalTok{(}\FunctionTok{if\_else}\NormalTok{(}\FunctionTok{is.na}\NormalTok{(}\FunctionTok{as.numeric}\NormalTok{(thall)),}\DecValTok{2}\NormalTok{,}\FunctionTok{as.numeric}\NormalTok{(thall)), }
                   \AttributeTok{levels =} \FunctionTok{c}\NormalTok{(}\DecValTok{1}\NormalTok{,}\DecValTok{2}\NormalTok{,}\DecValTok{3}\NormalTok{),}
                   \AttributeTok{labels =} \FunctionTok{c}\NormalTok{(}\StringTok{"fixed defect"}\NormalTok{,}\StringTok{"normal"}\NormalTok{,}\StringTok{"reversable defect"}\NormalTok{))}
\NormalTok{  )}
\end{Highlighting}
\end{Shaded}

\hypertarget{identifica-y-gestiona-los-valores-extremos.}{%
\subsection{Identifica y gestiona los valores
extremos.}\label{identifica-y-gestiona-los-valores-extremos.}}

Estudiaremos los boxplots de las variables numéricas para observar si
existen valores atípicos:

\begin{center}\includegraphics{TCVD_PRA2_files/figure-latex/unnamed-chunk-7-1} \end{center}

Vemos que todas las variables menos \emph{age} muestran valores que
están alejados de las medias y la mayoría de observaciones. Esta gráfica
identifica como atípico el valor que \(x < Q_1 - IQR\cdot 1.58\) o
\(x < Q_3 - IQR\cdot 1.58\) donde \(IQR = Q_3 - Q_1\). La manera en la
que vamos a tratarlos es asignarles el valor más cercano para que no
sean considerados \emph{outliers}.

\begin{Shaded}
\begin{Highlighting}[]
\ControlFlowTok{for}\NormalTok{ (col }\ControlFlowTok{in} \FunctionTok{colnames}\NormalTok{(data\_num))\{}
\NormalTok{  value }\OtherTok{=}\NormalTok{ data[[col]][data[[col]] }\SpecialCharTok{\%in\%} \FunctionTok{boxplot.stats}\NormalTok{(data[[col]])}\SpecialCharTok{$}\NormalTok{out]}
\NormalTok{  res }\OtherTok{\textless{}{-}} \FunctionTok{quantile}\NormalTok{(data[[col]], }\AttributeTok{probs =} \FunctionTok{c}\NormalTok{(}\DecValTok{0}\NormalTok{,}\FloatTok{0.25}\NormalTok{,}\FloatTok{0.5}\NormalTok{,}\FloatTok{0.75}\NormalTok{,}\DecValTok{1}\NormalTok{))}
\NormalTok{  q1 }\OtherTok{\textless{}{-}}\NormalTok{ res[[}\DecValTok{2}\NormalTok{]]}
\NormalTok{  q3 }\OtherTok{\textless{}{-}}\NormalTok{ res[[}\DecValTok{4}\NormalTok{]]}
\NormalTok{  iqr }\OtherTok{\textless{}{-}}\NormalTok{ q3 }\SpecialCharTok{{-}}\NormalTok{ q1}
\NormalTok{  min\_thshld }\OtherTok{\textless{}{-}}\NormalTok{ q1 }\SpecialCharTok{{-}} \FloatTok{1.58}\SpecialCharTok{*}\NormalTok{iqr}
\NormalTok{  max\_thshld }\OtherTok{\textless{}{-}}\NormalTok{ q3 }\SpecialCharTok{+} \FloatTok{1.58}\SpecialCharTok{*}\NormalTok{iqr}
\NormalTok{  data[[col]][data[[col]] }\SpecialCharTok{\textless{}}\NormalTok{ min\_thshld] }\OtherTok{=}\NormalTok{ min\_thshld}
\NormalTok{  data[[col]][data[[col]] }\SpecialCharTok{\textgreater{}}\NormalTok{ max\_thshld] }\OtherTok{=}\NormalTok{ max\_thshld}
\NormalTok{\}}


\NormalTok{data }\SpecialCharTok{\%\textgreater{}\%} 
  \FunctionTok{select\_if}\NormalTok{(is.numeric) }\SpecialCharTok{\%\textgreater{}\%} 
\FunctionTok{pivot\_longer}\NormalTok{(}\FunctionTok{colnames}\NormalTok{(data\_num)) }\SpecialCharTok{\%\textgreater{}\%} 
  \FunctionTok{as.data.frame}\NormalTok{() }\SpecialCharTok{\%\textgreater{}\%} 
  \FunctionTok{ggplot}\NormalTok{(}\FunctionTok{aes}\NormalTok{(}\AttributeTok{y =}\NormalTok{ value)) }\SpecialCharTok{+}    \CommentTok{\# Draw each column as histogram}
  \FunctionTok{geom\_boxplot}\NormalTok{(}\AttributeTok{fill =} \StringTok{"\#404080"}\NormalTok{, }\AttributeTok{coef =} \FloatTok{1.58}\NormalTok{) }\SpecialCharTok{+} 
  \FunctionTok{theme\_minimal}\NormalTok{() }\SpecialCharTok{+}
  \FunctionTok{facet\_wrap}\NormalTok{(}\SpecialCharTok{\textasciitilde{}}\NormalTok{ name, }\AttributeTok{scales =} \StringTok{"free"}\NormalTok{) }\SpecialCharTok{+}
  \FunctionTok{labs}\NormalTok{(}\AttributeTok{title=}\StringTok{"Boxplots de variables numéricas }\SpecialCharTok{\textbackslash{}n}\StringTok{ Outliers corregidos"}\NormalTok{)}
\end{Highlighting}
\end{Shaded}

\begin{center}\includegraphics{TCVD_PRA2_files/figure-latex/unnamed-chunk-8-1} \end{center}

\hypertarget{anuxe1lisis-de-los-datos}{%
\section{Análisis de los datos}\label{anuxe1lisis-de-los-datos}}

\hypertarget{selecciuxf3n-de-los-grupos-de-datos-que-se-quieren-analizarcomparar.}{%
\subsection{Selección de los grupos de datos que se quieren
analizar/comparar.}\label{selecciuxf3n-de-los-grupos-de-datos-que-se-quieren-analizarcomparar.}}

Lo primero que haremos será estudiar la correlación entre las variables
y también la que tienen con la variable \emph{output}.

\begin{center}\includegraphics{TCVD_PRA2_files/figure-latex/unnamed-chunk-9-1} \end{center}

Vemos que la variable \emph{output} no tiene correlación significativa
con \emph{fbs} y \emph{trtbps}, y con las que mayor correlación tiene
son \emph{caa}, \emph{oldpeak}, \emph{exng}, \emph{thallachh} y
\emph{cp}.

Respecto al resto de variables, vemos parejas relacionadas como
\emph{oldpeak} y \emph{slp}, \emph{cp} y \emph{exng}, \emph{age} y
\emph{thallachh}, \emph{thallachh} y \emph{slp}.

Por otro lado, resulta curioso observar el la variable \emph{chol} no se
relaciona más que con \emph{restecg}

\hypertarget{comprobaciuxf3n-de-la-normalidad-y-homogeneidad-de-la-varianza.}{%
\subsection{Comprobación de la normalidad y homogeneidad de la
varianza.}\label{comprobaciuxf3n-de-la-normalidad-y-homogeneidad-de-la-varianza.}}

Vamos a analizar las variables recién mencionadas, para determinar si
son aptas para aplicar tests de comparación de grupos.

\begin{Shaded}
\begin{Highlighting}[]
\NormalTok{data\_compare }\OtherTok{\textless{}{-}}\NormalTok{ data }\SpecialCharTok{\%\textgreater{}\%} 
  \FunctionTok{select}\NormalTok{(}
\NormalTok{    age,}
\NormalTok{    cp,}
\NormalTok{    fbs,}
\NormalTok{    thalachh,}
\NormalTok{    exng,}
\NormalTok{    oldpeak,}
\NormalTok{    slp,}
\NormalTok{    caa,}
\NormalTok{    output}
\NormalTok{  )}
\FunctionTok{head}\NormalTok{(data\_compare)}
\end{Highlighting}
\end{Shaded}

\begin{tabular}{r|l|l|r|l|r|l|l|l}
\hline
age & cp & fbs & thalachh & exng & oldpeak & slp & caa & output\\
\hline
63 & asymptomatic & TRUE & 150 & 0 & 2.3 & unsloping & 0 & more chance of heart attack\\
\hline
37 & non-anginal pain & FALSE & 187 & 0 & 3.5 & unsloping & 0 & more chance of heart attack\\
\hline
41 & atypical angina & FALSE & 172 & 0 & 1.4 & downsloping & 0 & more chance of heart attack\\
\hline
56 & atypical angina & FALSE & 178 & 0 & 0.8 & downsloping & 0 & more chance of heart attack\\
\hline
57 & typical angina & FALSE & 163 & 1 & 0.6 & downsloping & 0 & more chance of heart attack\\
\hline
57 & typical angina & FALSE & 148 & 0 & 0.4 & flat & 0 & more chance of heart attack\\
\hline
\end{tabular}

\hypertarget{normalidad}{%
\subsubsection{Normalidad}\label{normalidad}}

Vamos a analizar si las variables numéricas tienen distribuciones
normales. Para ello aplicamos el test \emph{Shapiro-Wilk}:

\begin{verbatim}
## 
##  Shapiro-Wilk normality test
## 
## data:  data_compare$oldpeak
## W = 0.85269, p-value = 2.615e-16
\end{verbatim}

\begin{verbatim}
## 
##  Shapiro-Wilk normality test
## 
## data:  data_compare$thalachh
## W = 0.97705, p-value = 9.187e-05
\end{verbatim}

\begin{verbatim}
## 
##  Shapiro-Wilk normality test
## 
## data:  data_compare$age
## W = 0.98664, p-value = 0.006745
\end{verbatim}

\begin{center}\includegraphics{TCVD_PRA2_files/figure-latex/unnamed-chunk-11-1} \end{center}

El test Shapiro-Wilk parte de la hipótesis de que la población está
normalmente distribuida, por lo que si el resultado del p-valor es mayor
que 0.05, podemos aceptar dicha hipótesis con un 95\% de confianza.

En el caso anterior, vemos que no se puede aceptar que tanto
\emph{oldpeak} como \emph{thalachh} estén normalmente distribuidas.

Podemos visualizar que las medias por los dos grupos objetivo no son
iguales:

\begin{center}\includegraphics{TCVD_PRA2_files/figure-latex/unnamed-chunk-12-1} \end{center}

\hypertarget{homogeneidad-de-la-varianza}{%
\subsubsection{Homogeneidad de la
varianza}\label{homogeneidad-de-la-varianza}}

Aplicando el test de Levene, analizamos si las varianzas de los grupos
de la variable objetivo son homogéneas para las tres variables numéricas
que estamos analizando:

\begin{Shaded}
\begin{Highlighting}[]
\FunctionTok{print}\NormalTok{(}\FunctionTok{leveneTest}\NormalTok{(oldpeak }\SpecialCharTok{\textasciitilde{}}\NormalTok{ output, }\AttributeTok{data =}\NormalTok{ data\_compare,))}
\end{Highlighting}
\end{Shaded}

\begin{verbatim}
## Levene's Test for Homogeneity of Variance (center = median)
##        Df F value    Pr(>F)    
## group   1  33.028 2.232e-08 ***
##       300                      
## ---
## Signif. codes:  0 '***' 0.001 '**' 0.01 '*' 0.05 '.' 0.1 ' ' 1
\end{verbatim}

\begin{Shaded}
\begin{Highlighting}[]
\FunctionTok{print}\NormalTok{(}\FunctionTok{leveneTest}\NormalTok{(thalachh }\SpecialCharTok{\textasciitilde{}}\NormalTok{ output, }\AttributeTok{data =}\NormalTok{ data\_compare,))}
\end{Highlighting}
\end{Shaded}

\begin{verbatim}
## Levene's Test for Homogeneity of Variance (center = median)
##        Df F value  Pr(>F)  
## group   1  5.0378 0.02553 *
##       300                  
## ---
## Signif. codes:  0 '***' 0.001 '**' 0.01 '*' 0.05 '.' 0.1 ' ' 1
\end{verbatim}

\begin{Shaded}
\begin{Highlighting}[]
\FunctionTok{print}\NormalTok{(}\FunctionTok{leveneTest}\NormalTok{(age }\SpecialCharTok{\textasciitilde{}}\NormalTok{ output, }\AttributeTok{data =}\NormalTok{ data\_compare,))}
\end{Highlighting}
\end{Shaded}

\begin{verbatim}
## Levene's Test for Homogeneity of Variance (center = median)
##        Df F value   Pr(>F)   
## group   1  7.6349 0.006079 **
##       300                    
## ---
## Signif. codes:  0 '***' 0.001 '**' 0.01 '*' 0.05 '.' 0.1 ' ' 1
\end{verbatim}

En los tres casos, obtenemos que \(p_{value} < 0.05\), por lo que
rechazamos la hipótesis nula y concluímos que ninguna tiene varianzas
homogeneas dentro de los grupos.

En resumen, ninguna de las variables analizadas cumplen con el supuesto
de normalidad y homogeneidad de varianza.

\hypertarget{aplicaciuxf3n-de-pruebas-estaduxedsticas-para-comparar-los-grupos-de-datos.}{%
\subsection{Aplicación de pruebas estadísticas para comparar los grupos
de
datos.}\label{aplicaciuxf3n-de-pruebas-estaduxedsticas-para-comparar-los-grupos-de-datos.}}

\hypertarget{prueba-de-chi-cuadrado}{%
\subsubsection{Prueba de chi-cuadrado}\label{prueba-de-chi-cuadrado}}

AHora vamos a analizar la correlación entre las variables categóricas y
la variable objetivo mediante el test \emph{Chi-cuadrado}, el cual
indica si existe correlación significativa entre dos variables
cualitativas.

\begin{Shaded}
\begin{Highlighting}[]
\CommentTok{\#categoricas Chi{-}quadrado}
\CommentTok{\# Variable "caa"}
\NormalTok{table\_caa }\OtherTok{\textless{}{-}} \FunctionTok{table}\NormalTok{(data\_compare}\SpecialCharTok{$}\NormalTok{caa, data\_compare}\SpecialCharTok{$}\NormalTok{output)}
\FunctionTok{print}\NormalTok{(}\FunctionTok{chisq.test}\NormalTok{(table\_caa))}
\end{Highlighting}
\end{Shaded}

\begin{verbatim}
## 
##  Pearson's Chi-squared test
## 
## data:  table_caa
## X-squared = 73.69, df = 4, p-value = 3.771e-15
\end{verbatim}

\begin{Shaded}
\begin{Highlighting}[]
\CommentTok{\# Variable "exng"}
\NormalTok{table\_exng }\OtherTok{\textless{}{-}} \FunctionTok{table}\NormalTok{(data\_compare}\SpecialCharTok{$}\NormalTok{exng, data\_compare}\SpecialCharTok{$}\NormalTok{output)}
\FunctionTok{print}\NormalTok{(}\FunctionTok{chisq.test}\NormalTok{(table\_exng))}
\end{Highlighting}
\end{Shaded}

\begin{verbatim}
## 
##  Pearson's Chi-squared test with Yates' continuity correction
## 
## data:  table_exng
## X-squared = 55.456, df = 1, p-value = 9.556e-14
\end{verbatim}

\begin{Shaded}
\begin{Highlighting}[]
\CommentTok{\# Variable "cp"}
\NormalTok{table\_cp }\OtherTok{\textless{}{-}} \FunctionTok{table}\NormalTok{(data\_compare}\SpecialCharTok{$}\NormalTok{cp, data\_compare}\SpecialCharTok{$}\NormalTok{output)}
\FunctionTok{print}\NormalTok{(}\FunctionTok{chisq.test}\NormalTok{(table\_cp))}
\end{Highlighting}
\end{Shaded}

\begin{verbatim}
## 
##  Pearson's Chi-squared test
## 
## data:  table_cp
## X-squared = 80.979, df = 3, p-value < 2.2e-16
\end{verbatim}

\begin{enumerate}
\def\labelenumi{\arabic{enumi}.}
\item
  Variable \texttt{cp}: El valor p es menor que 2.2e-16, que es menor
  que 0.05. Por lo tanto, rechazamos la hipótesis nula y concluimos que
  hay una asociación significativa entre las categorías de \texttt{cp}.
\item
  Variable \texttt{caa}: El valor p es 3.771e-15, que es menor que 0.05.
  Por lo tanto, rechazamos la hipótesis nula y concluimos que hay una
  asociación significativa entre las categorías de \texttt{caa}.
\item
  Variable \texttt{exng}: El valor p es 9.556e-14, que es menor que
  0.05. Por lo tanto, rechazamos la hipótesis nula y concluimos que hay
  una asociación significativa entre las categorías de \texttt{exng}.
\end{enumerate}

En resumen, todas las variables categóricas parecen tener una relación
significativa con la variable de resultado.

\hypertarget{modelo-de-regresiuxf3n-loguxedstica}{%
\subsubsection{Modelo de regresión
logística}\label{modelo-de-regresiuxf3n-loguxedstica}}

Al tratarse de una variable objetivo de dos clases, vamos ajustar un
modelo de regresión logística.

\begin{Shaded}
\begin{Highlighting}[]
\CommentTok{\# wilcox oldpeak}
\NormalTok{res }\OtherTok{\textless{}{-}} \FunctionTok{wilcox.test}\NormalTok{(oldpeak }\SpecialCharTok{\textasciitilde{}}\NormalTok{ output, }\AttributeTok{data =}\NormalTok{ data\_compare)}
\FunctionTok{print}\NormalTok{(res)}
\end{Highlighting}
\end{Shaded}

\begin{verbatim}
## 
##  Wilcoxon rank sum test with continuity correction
## 
## data:  oldpeak by output
## W = 16721, p-value = 3.397e-13
## alternative hypothesis: true location shift is not equal to 0
\end{verbatim}

\begin{Shaded}
\begin{Highlighting}[]
\CommentTok{\# kruskal thalachh}
\NormalTok{res }\OtherTok{\textless{}{-}} \FunctionTok{kruskal.test}\NormalTok{(thalachh }\SpecialCharTok{\textasciitilde{}}\NormalTok{ output, }\AttributeTok{data =}\NormalTok{ data\_compare)}
\FunctionTok{print}\NormalTok{(res)}
\end{Highlighting}
\end{Shaded}

\begin{verbatim}
## 
##  Kruskal-Wallis rank sum test
## 
## data:  thalachh by output
## Kruskal-Wallis chi-squared = 54.719, df = 1, p-value = 1.391e-13
\end{verbatim}

\begin{Shaded}
\begin{Highlighting}[]
\CommentTok{\# Regresión logística utilizando \textquotesingle{}cp\textquotesingle{}, \textquotesingle{}caa\textquotesingle{} y \textquotesingle{}exng\textquotesingle{} como predictores}
\NormalTok{model }\OtherTok{\textless{}{-}} \FunctionTok{glm}\NormalTok{(output }\SpecialCharTok{\textasciitilde{}}\NormalTok{ cp }\SpecialCharTok{+}\NormalTok{ caa }\SpecialCharTok{+}\NormalTok{ exng, }\AttributeTok{data =}\NormalTok{ data\_compare, }\AttributeTok{family =}\NormalTok{ binomial)}
\FunctionTok{summary}\NormalTok{(model)}
\end{Highlighting}
\end{Shaded}

\begin{verbatim}
## 
## Call:
## glm(formula = output ~ cp + caa + exng, family = binomial, data = data_compare)
## 
## Deviance Residuals: 
##     Min       1Q   Median       3Q      Max  
## -2.2751  -0.5887   0.3953   0.5236   2.3600  
## 
## Coefficients:
##                    Estimate Std. Error z value Pr(>|z|)    
## (Intercept)          0.6370     0.2973   2.143 0.032131 *  
## cpatypical angina    1.8384     0.4967   3.701 0.000215 ***
## cpnon-anginal pain   1.8728     0.3974   4.713 2.45e-06 ***
## cpasymptomatic       1.2807     0.5694   2.249 0.024493 *  
## caa1                -1.9302     0.3901  -4.947 7.52e-07 ***
## caa2                -2.3021     0.5238  -4.395 1.11e-05 ***
## caa3                -2.9253     0.7326  -3.993 6.52e-05 ***
## caa4                -0.1931     1.5149  -0.127 0.898573    
## exng1               -1.4281     0.3554  -4.018 5.86e-05 ***
## ---
## Signif. codes:  0 '***' 0.001 '**' 0.01 '*' 0.05 '.' 0.1 ' ' 1
## 
## (Dispersion parameter for binomial family taken to be 1)
## 
##     Null deviance: 416.42  on 301  degrees of freedom
## Residual deviance: 259.31  on 293  degrees of freedom
## AIC: 277.31
## 
## Number of Fisher Scoring iterations: 5
\end{verbatim}

\hypertarget{conclusiones}{%
\section*{Conclusiones}\label{conclusiones}}
\addcontentsline{toc}{section}{Conclusiones}

\hypertarget{refs}{}
\begin{CSLReferences}{1}{0}
\leavevmode\vadjust pre{\hypertarget{ref-manualUOC}{}}%
Laia Subirats Maté, Mireia Calvo González, Diego Oswaldo Pérez Trenard.
2019. \emph{Introducción a La Limpieza y Análisis de Los Datos}. FUOC.

\leavevmode\vadjust pre{\hypertarget{ref-ggplot2}{}}%
Ozdemir, Ozancan. 2022. \emph{An Introduction to Ggplot2}.

\leavevmode\vadjust pre{\hypertarget{ref-RMarkdownCookbook}{}}%
Xie, Dervieux, Yihui, and Emily Riederer. 2022. \emph{R Markdown
Cookbook}. Chapman \& Hall/CRC.

\leavevmode\vadjust pre{\hypertarget{ref-awesome_tables}{}}%
Zhu, Hao. 2019. \emph{Create Awesome LaTeX Table with Knitr::kable and
kableExtra}. Zhu, Hao.

\end{CSLReferences}

\end{document}
